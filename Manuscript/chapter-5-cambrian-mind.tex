\chapter{The Cambrian Mind}

\section{From Single Cells to Stories}

Explosions change what is possible. In biology, the Cambrian era produced new body plans; in consciousness, language produced new ways of being. The shift is not incremental but architectural—a reorganization that creates novel niches in which minds can live \parencite{varela1991embodied}.

\section{Cognitive Body Plans}

Symbolic thought functions like an evolutionary toolkit for cognition. Memory becomes archive, attention becomes spotlight, concepts become organs that metabolize experience. Linguistic scaffolds—categories, metaphors, and narratives—stabilize patterns the way skeletons stabilize bodies \parencite{deacon1997symbolic}.

\section{Constraints and Trade-offs}

Every specialization closes doors. As symbolic intelligence optimizes for abstraction and communication, immediacy and presence recede. The gains are vast—coordination, planning, culture—but the losses are real and measurable in lived texture and embodied fluency \parencite{raichle2001default,buckner2008brain}.

\section{Metaphor and Mechanism}

The Garden frame is a story; the brain is a mechanism. The two need not conflict. Metaphor can align with mechanism when we track functions to neural and developmental substrates, letting beauty point toward testable structure \parencite{damasio1999feeling}.

\bigskip
\noindent Bridge to Chapter 6. If this is the cambrian of mind, grammar is its gatekeeper—the rules that let forms arise and persist.
