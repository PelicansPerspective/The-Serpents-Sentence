\chapter{The Angel at the Gate is Grammar}

\section{Rules that Cut the World}

Grammar is the boundary-work of mind. It constrains and enables, turning raw potential into lawful form. Word order, agreement, deixis—simple levers that carve experience into composable units \parencite{tomasello2008origins}.

\section{The Power of Recursion}

With recursion, language becomes generative: finite rules yield infinite expression. This unlocks nested thought, counterfactuals, and planning across time—core features of the narrator self \parencite{deacon1997symbolic}.

\section{Costs of Constraint}

What grammar grants, it also limits. Preferred structures bias attention. Available constructions shape what we notice, remember, and can easily say. Sapir and Whorf warned that grammar doesn’t just package thoughts—it shapes them \parencite{sapir1929status,whorf1956language}.

\section{Bridging Description and Prescription}

We can study grammar empirically and still honor its role in lived meaning. The gate is angel and guard: it protects coherence while risking rigidity. Conscious use of form can reopen paths obscured by habit.

\bigskip
\noindent Bridge to Chapter 7. Once the gate is known, we cross into a sea where symbols swim—code, networks, and data defining a new ecology.
